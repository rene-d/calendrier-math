% latex 27.tex && dvisvgm -a -n 27.dvi

\documentclass{article}
\usepackage{tikz}
\usepackage[margin=0mm]{geometry}

\pagestyle{empty}
\geometry{left=0pt,top=0pt,paperwidth=90mm,paperheight=85mm}

\begin{document}

% l'interstice entre les pièces
\def\a{0.03}

% ┛ 180
% ┗ 90
% ┏ 0
% ┓ 270
\newcommand{\piece}[3]{
    %\draw [blue,shift={(#1+1,#2+1)}] (-1,-1) rectangle (1,1);
    %\draw [red,very thick,shift={(#1+1,#2+1)},rotate=#3] (-1,-1)--(0,-1)--(0,0)--(1,0)--(1,1)--(-1,1)--cycle;
    \fill [green,very thick,shift={(#1+1,#2+1)},rotate=#3] (-1+\a,-1+\a)--(-\a,-1+\a)--(-\a,\a)--(1-\a,\a)--(1-\a,1-\a)--(-1+\a,1-\a)--cycle;
}

\begin{tikzpicture}

    % le fond du damier
    \fill [red] (0,0) rectangle (8,8);

    % la rangée de bordure du damier (plusieurs dispositions possibles)
    \piece{0}{6}{0}
    \piece{0}{5}{180}
    \piece{0}{3}{270}
    \piece{0}{2}{90}

    \piece{0}{0}{90}
    \piece{1}{0}{270}
    \piece{3}{0}{90}
    \piece{4}{0}{270}

    \piece{6}{4}{0}
    \piece{6}{3}{180}
    \piece{6}{1}{270}
    \piece{6}{0}{90}

    \piece{2}{6}{90}
    \piece{3}{6}{270}
    \piece{5}{6}{90}
    \piece{6}{6}{270}

    % le carré interdit
    \draw [fill=yellow] (4+\a,3+\a) rectangle (5-\a,4-\a);

    % remplissage du centre en évitant le carré interdit
    \piece{2}{2}{90}
    \piece{4}{2}{180}
    \piece{2}{4}{0}
    \piece{3}{3}{0}
    \piece{4}{4}{270}

\end{tikzpicture}

\end{document}
